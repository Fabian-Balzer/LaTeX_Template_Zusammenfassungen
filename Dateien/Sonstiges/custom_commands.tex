
% Schönere Referenzen:
\let\oldhref\href
\renewcommand{\href}[2]{\oldhref{#1}{\underline{\bfseries#2}}} % Damit Links fett und unterstrichen werden


% Eine Antwort-Umgebung, in der alles verkehrt herum geschrieben wird:
\NewEnviron{Answer}
{%
\noindent
\rotatebox[origin=c]{180}{%
\noindent
\begin{minipage}[t]{\linewidth}
\BODY
\end{minipage}%
}%
}%


% Einige farbige Boxen für Beispiele, Definitionen und Sätze
% Definition
\definecolor{mygreen}{RGB}{17,100,8}
\newtcolorbox[auto counter,number within=section]{Def}[2][]{%
breakable, enhanced, sharp corners, rounded corners=northwest, rounded corners=southeast, colback=blue!5!white,colframe=black!75!black,fonttitle=\bfseries,
title=Definition~\thetcbcounter: #2,#1}
% Satz (mit weiterem geforderten Argument für die Satzbezeichnung wie z. B. Theorem)
\newtcolorbox[auto counter, number within=section]{Satz}[3][]{%
breakable, enhanced, sharp corners, rounded corners=southwest, rounded corners=northeast, colback=blue!5!white,colframe=red!75!black,fonttitle=\bfseries,
title=#2~\thetcbcounter: #3,#1}
% Beispiele
\newtcolorbox[auto counter,number within=section]{Beispiel}[2][]{%
breakable, enhanced, colback=blue!5!white,colframe=blue!75!black, fonttitle=\bfseries,
title=Beispiel~\thetcbcounter: #2,#1}
% Wiederholungen
\newtcolorbox[auto counter,number within=section]{Wiederholung}[2][]{%
breakable, enhanced, sharp corners, colback=green!5!white,colframe=mygreen!75!black, fonttitle=\bfseries,
title=Wiederholung~\thetcbcounter: #2,#1}


% Generelles Zeugs zum Aussehen:
\def\stretchint#1{\vcenter{\hbox{\stretchto[440]{\displaystyle\int}{#1}}}}
\def\bs{\mkern-12mu}
\pagestyle{scrheadings}
\clearpairofpagestyles


\numberwithin{equation}{section}
\let\oldsection\section  % Footnotecounter resetten
\renewcommand{\section}{\setcounter{footnote}{0}\oldsection}
\renewcommand{\thefootnote}{\Roman{footnote}}
\setlist[itemize]{topsep=1pt, itemsep=0pt, parsep=0.5pt}  % Itemize manipulieren
\setlist[enumerate]{topsep=1pt, itemsep=0pt, parsep=0.5pt}  % Itemize manipulieren


%%% Setzen der Code-Umgebung
\definecolor{mygreen}{rgb}{0,0.6,0}
\definecolor{mygray}{rgb}{0.5,0.5,0.5}
\definecolor{mymauve}{rgb}{0.7,0,0}

\lstset{ %
  backgroundcolor=\color{white},    % choose the background color
  basicstyle = \ttfamily,
  columns=fullflexible,
  breaklines=true,                  % automatic line breaking only at whitespace
  captionpos=b,                     % sets the caption-position to bottom
  commentstyle=\color{mygray},      % comment style
  frame=single,                     % Falls wir den Code einrahmen wollen
  language=python,                  % Wir benutzen nur Python
  showstringspaces=false,           % Damit keine komischen Leerzeichen angezeigt werden
  escapeinside={\%*}{*)},           % Für Latex im Code
  keywordstyle=\color{mygreen},     % keyword style
  stringstyle=\color{mymauve},      % string literal stylehttps://www.overleaf.com/project/5f50ba666b494e0001a2f9b9
}



%%%%%%%%%%% Titel, Profs, Zeitraum etc.
\newcommand{\titel}{Klausurvorbereitung WiSe 2020/2021\\
Einführung in die Geophysik}
\newcommand{\Kurztitel}{EGeo-Vorbereitung}
\newcommand{\Zeitraum}{SoSe 2021}
\newcommand{\Profs}{Zur Vorlesung von \textsf{Prof. Gajewski, Prof. Hort,\\Prof. Hübscher} und \textsf{Prof. Schippkus}}
\newcommand{\Autor}{Fabian Balzer}
\newcommand{\Version}{Version vom \today}
\title{\titel}
\date{\Zeitraum}
%%%%%%%%%%%