\section{Beispielboxen}
\begin{Beispiel}{Ganz tolles Beispiel}
So erzeugst du eine Beispielbox\footnote{Und sogar Fußnoten funktionieren! Wow.}.\\
\Zz{So würde ein Satz, der zu zeigen ist, aussehen.}\\
\Zb{Und hier kann man ihn beweisen.}
\end{Beispiel}


Schau dir am besten auch einige der nützlichen Mathe-Kommandos\footnote{und für richtig viel Spaß \href{\RickRollLink}{diesen Link hier}} an:
\begin{equation}
    M:=\Menge{x\in\mathbb{R}}{\Abs{x}<\BracedIn{100-\frac{1}{2}}}\implies \xvec_\tx{guter Text} = \Matrix{1&3\\2&0}
\end{equation}

\begin{lstlisting}
# Auch python-code kann genutzt werden
for i, my_obj in enumerate(["Asdf", "test", 100]):
    print(my_obj)
\end{lstlisting}
Und \code{my_obj in [100]} als Code geschrieben ist doch auch schön.